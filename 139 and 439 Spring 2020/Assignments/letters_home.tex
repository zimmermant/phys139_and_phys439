\documentclass[12pt, letterpaper]{article}
\usepackage[pdftex]{graphicx}
\usepackage{setspace}
\usepackage{tikz, tabularx, multicol}
\usepackage{tkz-euclide}
\usetikzlibrary{shapes,calc}
\usetkzobj{all}
\usepackage{ gensymb }
\usepackage{fourier}
\usepackage{amsmath, verbatim, ifpdf}
\usepackage{color}
\usepackage{enumerate, enumitem}
\usepackage{parskip}
\usepackage[cm]{fullpage}
\usepackage{hyperref}
\pagestyle{plain}
\parindent=0.00in

\begin{document}

\textbf{\Large Letters Home}

Over the course of the semester you will need to write three "Letters Home".  The letter (actually an email) is a conversational discussion of what you've accomplished in lab recently to someone without a physics background.  Your goal should be to teach the recipient a little of the physics related to what you've been up to.

The letter does not need to summarize everything you've done recently; it can focus on one aspect.  However, it should provide enough a good overview of most of what you've done since your last letter (or the start of the term if this is the first letter).  Your lab notebook should be your primary resource for writing this letter but you should also share a draft with your research team leader.

The goal of this activity is to give you practice in the following:
\begin{itemize}

\item Understanding your audience’s background and expectations, and incorporating those factors into decisions
about the writing style and content.

\item Providing adequate information.

 \item Employing graphics effectively.

\item Defining technical terms appropriately.

\item Writing with clarity, conciseness, and correctness.
\end{itemize}

The letter home assignment must satisfy the following criteria: 
\begin{enumerate}
\item You must select a recipient from “back home”; you are free to choose a different recipient for each activity, to repeat the same recipient for each letter, or to address it to multiple recipients. Most students select a relative, significant other, a friend, or a high school science teacher. 
\item You must CC or BCC me on the email at zimmermant@uwstout.edu  (This counts as you turning the assignment in).
\item You must write the letter home in the tone of an email. It should not sound like a lab report. I encourage you to write as you naturally would when emailing the recipient. 
\item You must describe the setup of the experiment in sufficient detail. 
\item The recipient must be able to learn some physics from the letter home. This criterion considers both the correctness of the information and the clarity of expression, placing a renewed emphasis on grammar and spelling. 
\item Finally, the letter home should summarize the results and conclusions of the experiment.  What is important here is the interpretation of the results, not the numbers or graphs themselves.
\end{enumerate}



\end{document}