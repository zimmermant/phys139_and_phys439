\documentclass[12pt, letterpaper]{article}
\usepackage[pdftex]{graphicx}
\usepackage{setspace}
\usepackage{tikz, tabularx, multicol}
\usepackage{tkz-euclide}
\usetikzlibrary{shapes,calc}
\usetkzobj{all}
\usepackage{ gensymb }
\usepackage{fourier}
\usepackage{amsmath, verbatim, ifpdf}
\usepackage{color}
\usepackage{enumerate, enumitem}
\usepackage{parskip}
\usepackage[cm]{fullpage}
\usepackage{hyperref}
\pagestyle{plain}
\parindent=0.00in

\begin{document}

\textbf{\Large Recommendations for the use of notebooks in upper-division physics lab courses}

Read the article \emph{Recommendations for the use of notebooks in upper-division physics lab courses} by Stanley and Lewandowski.  While reading, keep in mind that your goal is to come up with guidance for the PHYS-139 students that will make up your research teams.

Write up answers to the following questions:

\begin{enumerate}
%\item What was the research question the authors of this paper were trying to answer? 

%\item Why were they trying to answer this question (e.g. what is the big-picture problem they are interested in)?

%\item Briefly summarize what their methodology was for answering their research question.

%\item What are the strengths and weaknesses of their methodology?

\item List the main purposes of lab notebooks the researchers found.

\item What characteristics did all lab notebooks share?

\item The three principles that someone needs to keep in mind when writing a lab notebook are \emph{context}, \emph{audience}, and \emph{timescale}. 

\begin{enumerate}
	\item What does the term \emph{context} mean.  Give a couple of examples.

	\item What are the different audiences that you imagine will be reading your lab notebook entries in this class?

	\item How will thinking about these audiences influence what you write in your lab notebook?

	\item What are the different timescales that will be relevant to you and the PHYS-139 students in this class? 

\end{enumerate}


\item The article lists five types of information that can be included in a lab notebook.  They are \emph{objective information}, \emph{analytical information}, \emph{interpretive information}, \emph{sythesis information}, and \emph{brainstorming information}.  Below, give an example of each type that is different from what is mentioned in the article.

\begin{enumerate}
\item Give an example of \emph{objective information}

\item Give an example of \emph{analytical information}

\item Give an example of \emph{interpretive information}

\item Give an example of \emph{synthesis information}

\item Give an example of \emph{brainstorming information}

\end{enumerate}

\item Give an example of some type of information you might include in a lab notebook that doesn't fall into one of the five categories listed above.

\item What are the main difficulties the researchers found many scientists face?

\item The article groups the recommendation into three categories; instructor framing, lab activity design, and notebook evaluation.  In this context what does 'instructor framing' mean?

\item Give an example of how you would give purpose to keeping a lab notebook in this course.  Imagine you are in my shoes and are teaching the PHYS-139 students.

\item Give an example of what sort of format you would require PHYS-139 students to use.

\item Give an example of how you would encourage students to incorporate context, audience, and timescale into their lab notebook entries.

\item How would you encourage students to include different types of information in their lab notebooks and how would you tie this back to context, audience, and timescale?

\item How would you go about preparing students for the common difficulties they would encounter?

%\item Imagine you are teaching an introductory physics course and are going to have students keep lab notebooks.  What steps would you take to frame the use of lab notebooks as the instructor.  Write a few paragraphs and give this some thought.  You can ignore lab activity design and notebook evaluation and just focus on the framing.







\end{enumerate}





\end{document}