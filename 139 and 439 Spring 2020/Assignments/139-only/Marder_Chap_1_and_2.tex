\documentclass[12pt, letterpaper]{article}
\usepackage[pdftex]{graphicx}
\usepackage{setspace}
\usepackage{tikz, tabularx, multicol}
\usepackage{tkz-euclide}
\usetikzlibrary{shapes,calc}
\usetkzobj{all}
\usepackage{ gensymb }
\usepackage{fourier}
\usepackage{amsmath, verbatim, ifpdf}
\usepackage{color}
\usepackage{enumerate, enumitem}
\usepackage{parskip}
\usepackage[cm]{fullpage}
\usepackage{hyperref}
\pagestyle{plain}
\parindent=0.00in

\begin{document}

\textbf{\Large Research Methods in Science}

Read chapters 1 and section 2.1.1 - 2.1.4 in the Marder text \emph{Research Methods for Science} and answer the following questions.  You will turn them in at the start of class on Wednesday

\begin{enumerate}

\item The text lists six different research methods scientists engage in.  List each of the six methods and give a short one or two sentence description of each one.  Also, give an example (different from the text) of each type.

\item The text also lists allied areas of research (section 1.3.7).  Write a one paragraph explanation of which of those three you think is most useful and explain your reasoning.  Write a second paragraph about which of those three methods you think is least useful, again with reasoning.

\item What aspects are critical for a hypothesis-driven experiment?

\item What is the difference between a null hypothesis and an alternative hypothesis?

\item A classmate of your states "studying with music on makes it harder to retain what you studied".  What is the null hypothesis and alternative hypothesis for this statement?

\item How does the limitations of funding and time affect your ability to reduce random error in experiments?

\item You are given a 20-sided die.  You are asked to determine whether or not the die is fair (each face shows up with the same frequency).  Come up with a method to test this.  Explain what data you would take, how you would take it, and what factors you would need to consider.

\end{enumerate}




\end{document}