\documentclass[12pt, letterpaper]{article}
\usepackage[pdftex]{graphicx}
\usepackage{setspace}
\usepackage{tikz}
\usetikzlibrary{calc}
\usepackage{fourier}
\usepackage{amsmath}
\usepackage{enumerate}
\usepackage{parskip}
\usepackage{subcaption}
\usepackage[cm]{fullpage}
\usepackage{physics}
\usepackage{subcaption}

\pagestyle{plain}
\parindent=0.00in
\begin{document}


\section*{How to Read a Research Paper}


\textbf{Do not read the paper from start to finish - plan your approach to get the most information possible.}

\section*{Structure of a Research Paper}

Most papers have the following format: Introduction, Methods, Results, Discussion, Summary (IMRDS), although many articles will deviate from this to some small degree.  Theoretical papers will not include a Methods section and will have a series of derivations instead.

\begin{itemize}

  \item Introduction:  Why do we care about this problem, what related research have others done before, and what is the background information needed to understand the article.  The introduction serves to get you excited about the article and give you enough background to understand the article.  The introduction also frequently includes a review of the literature, which can be useful for finding further background information.

  \item Methods or Procedure: What experiment are they doing and how are they doing it?  The goal of the methods section is to give another researcher enough information so they could replicate the experiment.  This section tends not to be as important to you as a reader unless you are doing a similar experiment.

  \item Results:  What did the experimenters find?  This section will contain the interesting graphs and tables of data that resulted from the experiment.  This is the interesting part of the experiment.

  \item Discussion:  How doe these results fit into the bigger picture and what questions arose?  The discussion frequently compares the results to other experiments and sometimes points out issues that came up or questions that the results brought up.

  \item Conclusion or Summary: A short summary of the results.  Frequently this is a restatement of the abstract.
\end{itemize}

\section{Reading the Paper}

\textbf{Read critically - don't assume that everything is correct, look for limitations of the study, assumptions that limit the applicability of the results, what other experiments or results would make their argument more compelling.}

Take notes as you go.  Some people like to print out the paper and mark it up as they go while others will use a notebook or notes on a computer to jot down thoughts, questions, and ideas.  Don't assume you will remember everything you read after going through the entire paper - chances are you will read many papers on similar topics and they all start to run together quickly.

Write up a summary of the article that includes the research question, the key results, and any insights or points of interest to you.  When it comes time to write up your research results, you don't want to have to dive back into a stack of papers to find that one piece of evidence that supports your main argument.  When done, you should be able to answer the following questions:

\begin{enumerate}

  \item Why are you reading this article?  Don't read articles in the hopes that some nugget will jump out at you; there are too many articles to read for this to be productive.  Choose your articles wisely and have an end goal in mind (e.g. "I don't understand how a Fabry-Perot interferometer works so I'll find a paper that describes this in detail" or "the paper I just read referred to this article as containing useful background information that will help me understand their results").

  \item What is the research question they were trying to answer?  What answer did they come up with?  Did they answer their research question?

  \item What do the figures mean?  The most important element of a research paper is usually the graphs and figures.  These will tell you how the experiment is set up and what the results show.  Make sure you understand what the graphs are plotting and what features of the data are interesting.

  \item What does this symbol mean?  Equations are a compact way to write a lot of information so make sure you understand what the important equations say.  If you aren't sure what a specific variable means, read backwards through the paper until you find where the symbol is first used.  Frequently the paper will spell out what the symbol means at that point.

  \item What does this acronym mean?  As with symbols, understanding acronyms will help you make sense of a paper.  The first time an acronym is used the author will write out the words represented by the acronym so scan back through the paper to find the first occurrence (e.g. "the external cavity diode laser (ECDL) was used to create a magneto-optical trap (MOT).  The MOT consisted of ...").

  \item What is a bibliography for, anyway?  Pay close attention to the papers that are cited.  Frequently the results of a crucial paper will be summarized in a single sentence, so you want to identify which papers are related to the current paper and contain key pieces of information.

\end{enumerate}

Keep everything organized and document, document, document.  You will be parsing a lot of information so keeping all of your articles organized is crucial.  You also need to keep records of what you are reading so you will have that information handy.  Frequently I keep copies of papers in my online lab notebook and I have a page where I summarize all articles related to a given project (along with links to the articles in my notebook).  I also find it useful to copy the citation information for the article in case I need to use it later.

\section*{Things to Keep in Mind}

\begin{enumerate}

\item Always try to identify the research question

Read the abstract, introduction, and conclusion to see what question is being answered.  This is the 'big question' for the article.

\item Identify little questions that are asked and answered throughout the paper.

\item What research methods are being used?

Try to classify the research according to the methods listed in the Marder text:

\begin{itemize}

\item Test a hypothesis – Hypothesis-driven research
\item Measure a value – Experimental research (I)
\item Measure a function or relationship – Experimental research (II)
\item Construct a model – Theoretical research and applied math
\item Observational and exploratory research
\item Improve a product or process – Industrial and applied research
\end{itemize}


\end{enumerate}

\section*{How I Read}

This isn't the only way to go about reading but if you don't haven't developed a style yet, this is a good starting point.

\textbf{First pass:}  I read the abstract, introduction, and conclusion, then flip back to look at the figures.  I don't take notes, I'm just looking for the key points and to see if there is enough useful information to warrant a second pass.

\textbf{Second pass:} If the paper seems useful to me I take another read-through, taking notes this time.  I identify the key ideas, anything that strikes me as interesting or unusual, and write a list of terms and phrases I don't understand.  I will look up those things I don't understand either online (Google is your friend) or by looking up one of the cited sources.  I will pull up the key cited sources and do a first pass on them to get a better idea of the background.

\textbf{Deep dive:}  Once I feel confident I know the words and phrases I didn't understand on my second pass, I will sit down with the paper at go through it in earnest.  I will read through the entire paper and take thorough notes.  By the time I'm done I should be able to write a one paragraph summary of the article that I can refer to later.

\textbf{Revisiting the paper:}  If I've done a good job of summarizing the paper I shouldn't need to come back to it again.  The only time I would do this is if I'm trying to reproduce the derivation of something in the paper or when I'm citing the research in a paper I am writing.


\section*{Finding Useful Papers}

\begin{enumerate}
\item \textbf{Use bibliography of a paper}

Find which papers are associated with key ideas in the paper you start with.  Citation styles vary from journal to journal, but typically you will either see a superscripted number or a number in square brackets at the end of the sentence.  This number shows which article the ideas in the preceding sentence came from.  Look up the original paper and download it.

Repeat this process until you have a good idea of the key ideas behind your topic.  The sources cited in a paper will allow you to trace the research upstream to see what had been done before and to see what background you need to understand.  I call this following a trail up stream.

\item \textbf{Find papers that cite the paper you are currently reading}

If you go to Google Scholar and look up the paper you are currently reading, underneath the snippet of text Google shows you, you will see 'Cited by XXX' where 'XXX' is a number.  These are articles that referenced this paper in their bibliography.  You can follow the trail up stream by looking at these papers, as well as seeing what other papers they cited. 

\item \textbf{Keyword Search in Google Scholar}

If you have a good idea of some good key words to look for you can use Google Scholar to search for papers.  I typically only use this if I don't have any papers as a starting point or if I feel I've gone upstream and downstream from a paper I have and I'm not finding what I need.

\item \textbf{How to get papers}

Many papers are available for free via Google Scholar.  If you use Chrome I recommend getting the Unpaywall extension.  This will find any open-source versions of the article that might be available.  The next stop, if you haven't found the article yet, is to go the the UW-Stout library webpage.  Type the article title into the search box on the front page to see if it is available from the library.  If it isn't available directly from the library you can use the "Fetch it For Me" button on the top of the Stout library webpage to have the library get the article for you for free through interlibrary loan.  Interlibrary loan is your friend and I encourage you to take advantage of it.  

\end{enumerate}
 


\end{document}